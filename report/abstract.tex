% !Tex root = main.tex
\newpage
\section*{Abstract}
\addcontentsline{toc}{section}{Abstract}
\noindent Image reconstruction is a classical problem in computer vision.
While most of the state-of-the-art approaches for image reconstruction
focus on upscaling only, taking both down- and upscaling into account
offers the possibility to improve reconstruction performance while keeping the
image compression rate constant. Recent work has shown the feasibility of task
aware downscaling for the single-image-super-resolution and the image-colorization
task, focussing on the image domain only and excluding external effects.
\newline
Within this work advances task aware downscaling is advanced by analysing the
effect of perturbation and enhancing robustness and efficiency for super-resolution
and colorization in the image domain, by improving both the model architecture and
training procedure. As a result for the super-resolution task a model with $44 \%$
less parameters could be derived, while having a similar accuracy. For the
image-colorization task the reconstruction performance could be improved by
$31 \%$ with a slightly larger model. In addition, this work introduces a
training architecture to extends the task aware downscaling approach from the
image to the video domain.

% The abstract gives a concise overview of the work you have done. The reader shall be able to decide whether the work which has been done is interesting for him by reading the abstract. Provide a brief account on the following questions:

% \begin{itemize}
%  \item What is the problem you worked on? (Introduction)
%  \item How did you tackle the problem? (Materials and Methods)
%  \item What were your results and findings? (Results)
%  \item Why are your findings significant? (Conclusion)
% \end{itemize}

% \noindent The abstract should approximately cover half of a page, and does generally not contain citations.
